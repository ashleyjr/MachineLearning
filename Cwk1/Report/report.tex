%% ----------------------------------------------------------------
%% Article.tex
%% ---------------------------------------------------------------- 
\documentclass{ecsarticle}     % Use the Article Style
\graphicspath{{Figures/}}   % Location of your graphics files
\usepackage{natbib}            % Use Natbib style for the refs.
\hypersetup{colorlinks=false}   % Set to false for black/white printing
\input{Definitions}            % Include your abbreviations

\usepackage[nodayofweek]{datetime}
\usepackage{listings}
\usepackage{color}

%% ----------------------------------------------------------------
\begin{document}
\frontmatter
\title      {COMP6036: Advanced Machine Learning\\[1cm]
            An investigation into DBSCAN}
      
\addresses  {\deptname\\\univname}
\authors                 {\href{mailto:ajr2g10@ecs.soton.ac.uk}{Ashley J. Robinson}\\\href{mailto:ajr2g10@ecs.soton.ac.uk}{ajr2g10@ecs.soton.ac.uk}}

\date       {\today}
\subject    {}
\keywords   {}
\maketitle
%% ----------------------------------------------------------------



\begin{abstract}
Your abstract goes here
\end{abstract}

\mainmatter


\section{Introduction}

DBSCAN (Density-Based Spatial Clustering of Applications with Noise) is a clustering algorithm with three objectives~\citep{Ester96adensity-based}.

\begin{itemize}
   \item Minimise the required domain knowledge needed to set input parameters. 
   \item Have the capability to discover clusters of arbitrary shapes.
   \item Good performance on large spatial databases.
\end{itemize}


\section{Algorithm Motivation}
\section{Technical Explanation}
\section{Conclusion and Further Work}

\bibliographystyle{ecs}
\bibliography{references}



\end{document}
%% ----------------------------------------------------------------

